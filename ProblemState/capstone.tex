\documentclass[onecolumn, draftclsnofoot,10pt, compsoc]{IEEEtran}
\usepackage{graphicx}
\usepackage{url}
\usepackage{setspace}

\usepackage{geometry}
\geometry{textheight=9.5in, textwidth=7in}

% 1. Fill in these details
\def \CapstoneTeamName{			}
\def \CapstoneTeamNumber{		9}
\def \GroupMemberOne{			Nipun Bathini}
\def \GroupMemberTwo{			Carl Benson}
\def \GroupMemberThree{			Brandon Dring}
\def \CapstoneProjectName{		No more touch. No more Keyboard. Bring it All Together. Using Technology to Teach Humans}
\def \CapstoneSponsorCompany{	CDK Global}
\def \CapstoneSponsorPerson{	Trevor Moore}

% 2. Uncomment the appropriate line below so that the document type works
\def \DocType{		Problem Statement
				%Requirements Document
				%Technology Review
				%Design Document
				%Progress Report
				}
			
\newcommand{\NameSigPair}[1]{\par
\makebox[2.75in][r]{#1} \hfil 	\makebox[3.25in]{\makebox[2.25in]{\hrulefill} \hfill		\makebox[.75in]{\hrulefill}}
\par\vspace{-12pt} \textit{\tiny\noindent
\makebox[2.75in]{} \hfil		\makebox[3.25in]{\makebox[2.25in][r]{Signature} \hfill	\makebox[.75in][r]{Date}}}}
% 3. If the document is not to be signed, uncomment the RENEWcommand below
\renewcommand{\NameSigPair}[1]{#1}

%%%%%%%%%%%%%%%%%%%%%%%%%%%%%%%%%%%%%%%
\begin{document}
\begin{titlepage}
    \pagenumbering{gobble}
    \begin{singlespace}
    	%\includegraphics[height=4cm]{coe_v_spot1}
        \hfill 
        % 4. If you have a logo, use this includegraphics command to put it on the coversheet.
        %\includegraphics[height=4cm]{CompanyLogo}   
        \par\vspace{.2in}
        \centering
        \scshape{
            \huge CS Capstone \DocType \par
            {\large\today}\par
			{\large CS461 Fall 2017}\par
            \vspace{.5in}
            \textbf{\Huge\CapstoneProjectName}\par
            \vfill
            {\large Prepared for}\par
            \Huge \CapstoneSponsorCompany\par
            \vspace{5pt}
            {\Large\NameSigPair{\CapstoneSponsorPerson}\par}
            {\large Prepared by }\par
            Group\CapstoneTeamNumber\par
            % 5. comment out the line below this one if you do not wish to name your team
            \CapstoneTeamName\par 
            \vspace{5pt}
            {\Large
                \NameSigPair{\GroupMemberOne}\par
                %\NameSigPair{\GroupMemberTwo}\par
                %\NameSigPair{\GroupMemberThree}\par
            }
            \vspace{20pt}
        }
        \begin{abstract}
        % 6. Fill in your abstract 
			This document will be outlining the problem statement for the integration of personal assistants, like Amazon's Amazon Echo, 
			with the new up and coming Virtual Reality headsets. CDK global is an automotive software company, and works closely with dealers. The integration
			will allow the user to wear the head set and ask the personal assistant ``Hey Alexa, tell me Subaru's sales data for last Monday''. The virtual headset would then 
			be populated with interactive data for the user. In addition to the two devices, a third wearable smart watch or band. The wearable smart watch will be able to
			monitor heart rate. Monitoring heart rate, can let the devices know how the user is reacting to the data.
			
			This problem can be solved by obtaining a personal assistant, virtual reality headset, and a smart band/watch. All three devices must be able to send data to each other.
			Having this integration will give users the ability to obtain all the data they seek, by simply using their voice. 
        \end{abstract}     
    \end{singlespace}
\end{titlepage}
\newpage
\pagenumbering{arabic}
\tableofcontents
% 7. uncomment this (if applicable). Consider adding a page break.
%\listoffigures
%\listoftables
\clearpage

% 8. now you write!
\section{Definition and Description of Problem}
	In 2014, Amazon released one of their newest creations, Amazon Echo, a soon to be popular virtual assistant. The echo is a cylinder shaped device that is voice activated and is
	ready for the users commands. Alexa, the name the echo goes by, will tell you the weather, but can also set you a timer or alarm. Another up and coming device is the virtual reality
	headset. The days being able to wear a device on your head, that will place you in a different world are coming near. These devices, as the name implies, allows users to wear a headset 
	that includes a screen, and the screen will move and change as the user moves and commands the device. A third rising device is the smart watch or smart band. These devices are mostly
	popular for their ability to sync almost perfectly with smart phones. Giving us access to text messages, phone calls, emails and much more all on our wrist. Heart rate can also
	be obtained from this wrist accessory. The beauty of many technologies, is that they all can be linked together, so why not link these three personal devices together?
	
	CDK Global, is an automotive software company, working closely with dealerships, helping them with software for not only car sales, but also parts and more. Currently, the users of CDK software
	view all their data on boring computer screens. Although, current software works, with new assistant technologies, we could make receiving data more efficient of the users. Data can be connected
	to Alexa, so that if a dealer asks ``What were the Dodge sales for September 10th, 2017?''. Alexa will then respond with sales for that date. Much simpler than, having to open up a computer and look
	up the sales information. One step further would be connecting the data Alexa provides to a virtual reality headset. Wear the headset, and the user can ask the same question to Alexa. That data can 
	then be presented to screen, and allow the user to move their body around to control the screen. Adding controllers with the virtual headset, may give users an easier experience with going through the 
	data. To further enhance the experience, the addition of a smart band to receive the user’s heart rate.  Using the user’s heart rate, using machine learning, the device can learn how users react to certain 
	data. The band will automatically record the response and can label or alert the user. All in all, when all three devices are connected, the user is provided with a more engaging and efficient method of receiving
	and reporting data, when comparing the alternative, sitting at a desk with a mouse and keyboard. 
	
	

\section{Proposed Solution}
	In order for our team to introduce a new revolutionizing way of browsing through data, that will replace the use of manually browsing through a computer, or flipping through papers. Our team must first learn the ins 
	and outs of the three main devices. For the echo, we need to understand how the echo itself retrieves the data. Alexa must be able to access the specific dealerships database, and temporarily store that data. A better
	understanding of how the data is stored in the echo, will help us in the next step of sending the data to the virtual headset. For the virtual headset, we must design a layout for data. The layout must be in a way that
	is easy to the user’s eyes, and could even differ from user to user. The team must learn how motion is controlled on the headsets, to allow our users to interact with the data they are presented. For example, one feature
	could be converting data on screen into graphs. The team also has to go into depth, on how smart bands retrieve heart rate data. An understanding of how the data is obtained and stored, will allow us to connect the smart 
	bands data to the virtual headset. If the watch records a jump in the heart rate, an alert can be displayed on the headset, asking the user if they want to take further action with the data they are observing. Since the
	data will be interactive, the user is bound to want to make changes, add notes, save graphs and more. For the parts department, if they notice a shortage of a particular part, they will be able to place an order all through 
	the headset. In order to do this, we need to connect the virtual headset back to the database.  Overall, using these three devices will make the process more efficient for dealers, because they receive the data they need 
	faster and with less effort. Furthermore, the linking of these technologies can be expanded to be used outside of the automotive world.
	
	We will know we have completed our project, when we have obtained a personal assistant, virtual reality headset and smart band. When we ask the assistant for sales data for a particular date and company, the assistant will
	then send the data to the headset. Through the headset we will be able to organize to data to our personal preferences. Data that is unexpected, will result in an alert from the smart band to the headset.

	
\end{document}





